\documentclass[12pt]{article}
\usepackage[utf8]{inputenc}
\usepackage{amsmath}
\usepackage{graphicx}
\usepackage{float}
\usepackage{caption}
\usepackage{appendix}
\usepackage{gensymb}
\usepackage{pdfpages}

\newcommand{\figuremine}[2]{
\begin{figure}[H]
\noindent\makebox[\textwidth]{
 \includegraphics[width=1.4\textwidth]{#1}}
 \caption{#2}
\end{figure}
}

\setcounter{tocdepth}{3}
\title{ChalkBot Final Report}
\author{George Karavaev\\Alex Suchko\\Project\#1}

\begin{document}
  \maketitle 
 \abstract{ Chalkbot is a robot that is capable of drawing on  masonry surfaces with off the shelf chalk. Function is to easily do advertisments with chalk.}
\\
\\
{\centering
\includegraphics[width=0.6\textwidth]{../chalk_bot_logo2.png}
}
\newpage
 \tableofcontents
\listoffigures
\newpage
 \section{Introduction}
\subsection{Purpose}
 Benefits to a customer of our product at end of semester include:
 \begin{itemize}
  \item Energy efficiency
  \item Motor Control
  \item On-board Linux based computing
  \item Dedicated low-level motion processor
  \item Power supply monitor/minder
\end{itemize}

\subsection{Project Functions}
Features of our product at end of semester include:
   \begin{itemize}
  \item Two channels of switch-mode power supply for power efficient operation of power hungry components
  \item Two channels of full-bridge motor control, with synchronous rectification to reduce diode dissipation
  \item Two channels of high-frequency quadrature encoder feedback
  \item Control of chalk mechanism actuators
  \item Inertial yaw rate sensing (MEMS gyro)
  \item Extensive self diagnostic sensing, multiple point monitoring of voltage, current, power production/consumption, and temperature of key system components
  \item Attention to thermal design of high performance components to enable effective passive cooling
  \item Uses economical and environmentally friendly sidewalk chalk instead of a spray or slurry chalk, distinct from other sidewalk drawing machines that can be found on video sharing sites
\end{itemize}


\subsection{Blocks of work}
 Our project includes three major catetories. 
 \begin{description}
  \item{\bf HBridge}. Used to drive the main motors of the robot. 
  \item{\bf Power Supply}. Used to convert battery voltage to stable 5 volts.
  \item{\bf Carrier Board}. Used to connect modules together. Contains a number of sensors and proccesors for other necessary functions.
\end{description}

\section{Design}
\subsection{Design Alternatives}
The biggest alternatives we had was regarding the chalk mechanism. Our decision to use off the shelf chalk made it easiest to develop a fixed-position chalk mechanism with the robot drawing as it moves. 
\subsection{General Block Diagram}
\figuremine{../Proposal/Block_Diagram.png}{Overall Diagram}

\subsection{Detailed desription of design}
 \subsubsection{Hbridge}

\subsubsection{Switching Power Supply}


\subsubsection{Carrier Board}
\subsection{Schematics}
 \begin{description}
  \item{\bf HBridge}. For detailed schematics of the HBridge please refer to Appendix A, part 2. 
  \item{\bf Power Supply}. For detailed schematics of the Power Supply please refer to Appendix B, part 2.
  \item{\bf Carrier Board}. For detailed schematics of the Carrier Board please refer to Appendix C, part 2.
\end{description}


\section{Verification}
\subsection{Testing Procedure}
\subsection{Functional Testing}
\subsection{Quantitative Results}

\section{Costs}
\subsection{List of Equipment used}
\begin{itemize}
\item{\bf Oscilloscope}
\item{\bf Power Supply}
\item{\bf MultiMeter}
\item{\bf Waveform Generator}
\item{\bf Power resistors}
\item{\bf Test Motors}
\item{\bf Batteries}
\item{\bf Soldering Iron}
\item{\bf Microscope}
\item{\bf IR temperature sensor}
\item{\bf Teensy 2.0+}
\end{itemize}
\subsection{Bill of Materials}
\subsection{Total Cost}

\section{Conclusion}
\subsection{Accomplishments}
\subsection{Uncertainties}
\subsection{Ethics}
\subsection{Future Work}

\newpage
\appendix
\addappheadtotoc
\includepdf[pages={-}]{./dtsheets/Switch.pdf}
\newpage
\includepdf[pages={-}]{./dtsheets/Hbridge.pdf}
\newpage

 \end{document}
